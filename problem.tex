\chapter{Descripción del problema}
\label{chap:problem}
El cualquier comunidad con residentes existen problemas organizacionales y de carácter social, por una parte es un gran problema organizar a gran cantidad de personas para llegar a un acuerdo ya sea Reunion o entregar información relevante.\\
También existen necesidades en cuanto a la seguridad ya sea en cuanto a tecnología de las herramientas utilizadas como la eficiencia de estas. A continuación se presentan los problemas a solucionar con este proyecto en base a lo ya mencionado.
\section{Comunicación e información}
Problemas relacionados con la débil comunicación entre residentes ya sea por su gran cantidad, diseño del edificio, falta de espacios, etc.\\
A partir de esto también surgen los problemas de organización entre ellos mismos que de ser resueltos brinda una mejor experiencia en el inmueble compartido.
\subsection{Comunicación}
Existen diversas situaciones en las cuales la comunidad de residentes requiere llegar a acuerdos ya sea fechas, acuerdos, reuniones o dar avisos relevantes.\\
Hoy en día la gran mayoría de comunidades solo intercambia información por anuncios en algún espacio a la vista de todos o con un mensaje al conserje, acción que no es eficiente.
\subsection{Gastos comunes}
El sistema de gastos comunes se maneja en casi la totalidad de los casos por una persona. Esta debe velar por mantener las cuentas al día, pero es un trabajo arduo orginzacionalmente hablando. Sus principales dificultades son:
\begin{itemize}
	\item Realizar contabilidad
	\item Cobranza a vecinos
	\item Solucionar morosidades
	\item Coordinar formas de pago 
\end{itemize}
\subsection{Espacios compartidos}
La gran cantidad de residentes y/o la unicidad en el tipo de espacio compartido, generan colisiones en las peticiones de residentes al momento de su uso.\\
El poco acceso al listado de solicitudes de espacios compartidos le generan al usuario problemas de coordinación a la hora de solicitar uno.
\section{Seguridad}
El aumento de la delincuencia en nuestro país, ha dejado en evidencia las falencias de los sistemas de seguridad de inmobiliaria, este proyecto tiene como objetivo otorgar a la comunidad monitoreo y registro de información, para así poder tener medidas preventivas o fácil acceso a información histórica en caso de investigación.\\ 
\subsection{Conserjería}
Los servicios de conserjería comunes constan de un libro de visitas y un conserje, esto genera:
\begin{itemize}
	\item Mayor probabilidad de error del encargado.
	\item Dudas en investigación en caso de delitos.
	\item Visitas no recibidas son poco notificadas a los residentes.
	\item Difícil acceso a un registro histórico de visitas.
\end{itemize} 
\subsection{Estacionamientos}
Las cámaras de vigilancia de estacionamiento solo están disponibles a guardias o conserjes.\\
Tampoco existe un método de como saber la disponibilidad de los estacionamientos de visita de cada propietario.


