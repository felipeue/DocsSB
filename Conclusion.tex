\chapter{Conclusión}
\label{chap:conclusion}

La Ley Nº19223 promulgada por el Congreso Nacional en 1993 e impulsada por el diputado José Antonio Viera Gallo en el año 1992 abarca dos grandes figuras penales como lo son: espionaje y sabotaje (ambos aplicados a los sistemas de información). Sin embargo, existen casos en que las leyes establecidas por el poder legislativo no son efectivas y pueden contemplar un perjuicio tanto para la víctima como para el victimario. Por ejemplo, la \textit{Ley de transparencia y acceso a la información pública} hizo que aquellas plataformas dependientes del estado de mostraran información sensible acerca de todos los chilenos, i.e, registro civil; estando la información disponible para cualquier persona que disponga de ella. Lo anterior si bien es cierto no constituye la comisión de un delito puede representar la acción inicial para una suplantación de identidad.


Por otro lado, cómo se puede entender en el estudio descrito a lo largo de este documento la ley Nº19223 abarca todos los posibles ataques que se pueden realizar a sistemas de información o al menos las categorías mas importantes de ataques actualmente existentes se encuentran cubiertas por la ley. Es por esto que afirmamos que la ley es lo suficientemente robusta ante los distintos tipos de ataque.


%        - Deficiencias de la ley. 
%        - Dar una opinión sobre el estado de la ley, comentar si ésta es lo suficientemente robusta para los tipos de actuales.
%    - Si existen vacíos legales en la Ley explicarlos y dar sugerencia de que artículos se podrían agregar para cubrir los casos
